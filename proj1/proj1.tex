%!TEX builder = latexmk
%!TEX program = xelatex
\documentclass{article}
\usepackage{enumitem}
\setenumerate[1]{leftmargin=2em, itemsep=0pt, partopsep=0pt, parsep=\parskip, topsep=0pt}
\setenumerate[2]{leftmargin=2em, itemsep=0pt, partopsep=0pt, parsep=\parskip, topsep=0pt}
\setitemize[1]{leftmargin=2em, itemsep=0pt, partopsep=0pt, parsep=\parskip, topsep=0pt}
\setitemize[2]{leftmargin=2em, itemsep=0pt, partopsep=0pt, parsep=\parskip, topsep=0pt}
\usepackage{amsmath,amsfonts,amsthm,amssymb}
\usepackage{setspace}
\usepackage{fancyhdr}
\usepackage{lastpage}
\usepackage{extramarks}
\usepackage{chngpage}
\usepackage{soul,color}
\usepackage{graphicx,float,wrapfig}
\usepackage{CJK}
\usepackage{algorithm}  
\usepackage{algpseudocode} 
\usepackage{longtable}
\usepackage{listings}
\usepackage{color}
\usepackage{extarrows}
\usepackage{subfigure}
\usepackage{indentfirst}
\newcommand{\Class}{Operating System}
\newcommand{\ClassInstructor}{Wei Xu}

% Homework Specific Information. Change it to your own
\newcommand{\Title}{Project 1}
\newcommand{\DueDate}{Oct 29, 2020}
\newcommand{\StudentName}{}
\newcommand{\StudentClass}{}
\newcommand{\StudentNumber}{}

% In case you need to adjust margins:
\topmargin=-0.45in      %
\evensidemargin=0in     %
\oddsidemargin=0in      %
\textwidth=6.5in        %
\textheight=9.0in       %
\headsep=0.25in         %

% Setup the header and footer
\pagestyle{fancy}                                                       %
\lhead{\StudentName}                                                 %
\chead{\Title}  %
\rhead{\firstxmark}                                                     %
\lfoot{\lastxmark}                                                      %
\cfoot{}                                                                %
\rfoot{Page\ \thepage\ of\ \protect\pageref{LastPage}}                          %
\renewcommand\headrulewidth{0.4pt}                                      %
\renewcommand\footrulewidth{0.4pt}                                      %

%%%%%%%%%%%%%%%%%%%%%%%%%%%%%%%%%%%%%%%%%%%%%%%%%%%%%%%%%%%%%
% Some tools
\newcommand{\enterProblemHeader}[1]{\nobreak\extramarks{#1}{#1 continued on next page\ldots}\nobreak%
                                    \nobreak\extramarks{#1 (continued)}{#1 continued on next page\ldots}\nobreak}%
\newcommand{\exitProblemHeader}[1]{\nobreak\extramarks{#1 (continued)}{#1 continued on next page\ldots}\nobreak%
                                   \nobreak\extramarks{#1}{}\nobreak}%

\newcommand{\homeworkProblemName}{}%
\newcounter{homeworkProblemCounter}%
\newenvironment{homeworkProblem}[1][Problem \arabic{homeworkProblemCounter}]%
  {\stepcounter{homeworkProblemCounter}%
   \renewcommand{\homeworkProblemName}{#1}%
   \section*{\homeworkProblemName}%
   \enterProblemHeader{\homeworkProblemName}}%
  {\exitProblemHeader{\homeworkProblemName}}%

\newcommand{\homeworkSectionName}{}%
\newlength{\homeworkSectionLabelLength}{}%
\newenvironment{homeworkSection}[1]%
  {% We put this space here to make sure we're not connected to the above.

   \renewcommand{\homeworkSectionName}{#1}%
   \settowidth{\homeworkSectionLabelLength}{\homeworkSectionName}%
   \addtolength{\homeworkSectionLabelLength}{0.25in}%
   \changetext{}{-\homeworkSectionLabelLength}{}{}{}%
   \subsection*{\homeworkSectionName}%
   \enterProblemHeader{\homeworkProblemName\ [\homeworkSectionName]}}%
  {\enterProblemHeader{\homeworkProblemName}%

   % We put the blank space above in order to make sure this margin
   % change doesn't happen too soon.
   \changetext{}{+\homeworkSectionLabelLength}{}{}{}}%

\newcommand{\Answer}{\textbf{Answer:} }
\newcommand{\Acknowledgement}[1]{\ \\{\bf Acknowledgement:} #1}

%%%%%%%%%%%%%%%%%%%%%%%%%%%%%%%%%%%%%%%%%%%%%%%%%%%%%%%%%%%%%


%%%%%%%%%%%%%%%%%%%%%%%%%%%%%%%%%%%%%%%%%%%%%%%%%%%%%%%%%%%%%
% Make title
\title{\textmd{\bf \Class: \Title}\\{\large Instructed by \textit{\ClassInstructor}}\\\normalsize\vspace{0.1in}\small{Due\ on\ \DueDate}}
\date{}
\author{\textbf{\StudentName}\ \ \StudentClass\ \ \StudentNumber}
%%%%%%%%%%%%%%%%%%%%%%%%%%%%%%%%%%%%%%%%%%%%%%%%%%%%%%%%%%%%%


%%%%%%%%%%%%%%%%%%%%%%%%%%%%%%%%%%%%%%%%%%%%%%%%%%%%%%%%%%%%%
% Listing Settings
\definecolor{mygreen}{rgb}{0,0.6,0}
\definecolor{mygray}{rgb}{0.5,0.5,0.5}
\definecolor{mymauve}{rgb}{0.58,0,0.82}

\lstset{
  aboveskip=1em,                   % above skip space
  backgroundcolor=\color[rgb]{0.9,0.9,0.9},
                                   % choose the background color; you must add \usepackage{color} or \usepackage{xcolor}; should come as last argument
  basicstyle=\ttfamily,            % the size of the fonts that are used for the code
  breakatwhitespace=false,         % sets if automatic breaks should only happen at whitespace
  breaklines=true,                 % sets automatic line breaking
  captionpos=b,                    % sets the caption-position to bottom
  commentstyle=\color{mygreen},    % comment style
  deletekeywords={...},            % if you want to delete keywords from the given language
  escapeinside={\%*}{*)},          % if you want to add LaTeX within your code
  extendedchars=true,              % lets you use non-ASCII characters; for 8-bits encodings only, does not work with UTF-8
  frame=single,                    % adds a frame around the code
  keepspaces=true,                 % keeps spaces in text, useful for keeping indentation of code (possibly needs columns=flexible)
  keywordstyle=\color{blue},       % keyword style
  morekeywords={algexpr, frac, sqrt, pwr, b1, b2, b3, ln, sin, term},  
                                   % if you want to add more keywords to the set
  numbers=left,                    % where to put the line-numbers; possible values are (none, left, right)
  numbersep=5pt,                   % how far the line-numbers are from the code
  numberstyle=\color{mygray},      % the style that is used for the line-numbers
  rulecolor=\color{black},         % if not set, the frame-color may be changed on line-breaks within not-black text (e.g. comments (green here))
  showspaces=false,                % show spaces everywhere adding particular underscores; it overrides 'showstringspaces'
  showstringspaces=false,          % underline spaces within strings only
  showtabs=false,                  % show tabs within strings adding particular underscores
  stepnumber=5,                    % the step between two line-numbers. If it's 1, each line will be numbered
  stringstyle=\color{mymauve},     % string literal style
  tabsize=2,                       % sets default tabsize to 2 spaces
  title=\lstname,                  % show the filename of files included with \lstinputlisting; also try caption instead of title
  xleftmargin=2em,                 % left margin
  xrightmargin=2em,                % right margin
}
%%%%%%%%%%%%%%%%%%%%%%%%%%%%%%%%%%%%%%%%%%%%%%%%%%%%%%%%%%%%%


\renewcommand{\algorithmicrequire}{\textbf{Input:}}  % Use Input in the format of Algorithm  
\renewcommand{\algorithmicensure}{\textbf{Output:}} % Use Output in the format of Algorithm  


\newcommand*{\dif}{\mathop{}\!\mathrm{d}}
\newcommand*{\img}{\mathrm{i}}
\newcommand*{\dps}{\displaystyle}
\newcommand*{\lf}{\left\lfloor}
\newcommand*{\rf}{\right\rfloor}
\newcommand*{\lc}{\left\lceil}
\newcommand*{\rc}{\right\rceil}
\newcommand*{\ovt}[2]{\dps{\mathop{#1}^{#2}}}

\newtheorem{lma}{Lemma}

\newcommand{\lm}[1]{\textbf{Lemma \ref{#1}}}
\newcommand{\lgd}[2]{\left(\frac{#1}{#2}\right)}
\newcommand{\ds}[2]{\dps \frac{\partial #1}{\partial #2}}
\newcommand{\bds}[2]{\dps \frac{\partial}{\partial #2}\left(#1\right)}
\newcommand{\Eds}[2]{\dps \frac{\partial^2 #1}{\partial #2^2}}
\newcommand{\eds}[3]{\dps \frac{\partial^2 #1}{\partial #2\partial #3}}
\newcommand{\E}[1]{\mathbb{E}\left[#1\right]}
\newcommand{\Var}[1]{\mathrm{Var}\left[#1\right]}
\newcommand{\Cov}[1]{\mathrm{Cov}\left[#1\right]}

\newcommand{\Fig}[1]{\textbf{Figure \ref{#1}}}

\begin{document}
\begin{spacing}{1.1}
\maketitle \thispagestyle{empty}
%\cite{}
%%%%%%%%%%%%%%%%%%%%%%%%%%%%%%%%%%%%%%%%%%%%%%%%%%%%%%%%%%%%%
% Begin edit from here

\begin{homeworkProblem}[Task I]

\texttt{join()} and \texttt{finish()} are modified. A \texttt{swap} thread is used to record the current thread in 
\texttt{join()}, which is suspended. When a thread calls \texttt{finish()}, it makes its \texttt{swap} thread ready.

\texttt{JoinTest()} is the test function, and it is called inside \texttt{KThread.selfTest()}.
It creates three threads $X,Y,Z$.
$Z$ runs first, then $X$, finally $Y$. $X$ joins both $Z$ and $Y$, but $Y$ starts later than $X$. So this function tests
the two condition of \texttt{join()}.

For task I through IV, all the test functions are called in \texttt{ThreadedKernel.selfTest()}.

\end{homeworkProblem}

\begin{homeworkProblem}[Task II]

Test (\texttt{Condition2.selfTest()}) is a Consumer-Producer problem,
where 5 consumers and 5 producers are ordered in a random manner.
We need to check whether there is no item left in the end.

\end{homeworkProblem}

\begin{homeworkProblem}[Task III]

In the test (\texttt{Alarm.selfTest()}), 5 threads are created and each of them waits for a random time.
A random time interval is placed between contiguous threads. The purpose of the test is to check whether
some threads do not get enough wait time or wait for too long.

\end{homeworkProblem}

\begin{homeworkProblem}[Task IV]

Test (\texttt{Communicator.selfTest()}) is a modified Consumer-Producer problem,
where each item is a random number. We need to make sure that all Listeners get exactly one word and all the words
they get are different.

\end{homeworkProblem}

\begin{homeworkProblem}[Task V]

Test (\texttt{Lock.schedulerTest}) is implemented in \texttt{Lock.java}, where we construct 9 threads which are waiting for a lock that is held by a thread with relatively low priority. Meanwhile, a thread with higher priority consumes CPU time as long as the lock is not free and it revokes the $yield$ function periodically. Without priority donation, thread 0 will be stuck forever. 

Test (\texttt{Condition2.schedulerTest}) relies on the Consumer-Producer model to test whether the queue is correctly organized when the number of consumers get larger. 

In both of the test, we print the information of the queue at some critical points. 

\end{homeworkProblem}

\begin{homeworkProblem}[Task VI]
  Only \texttt{Boat.java} needs to be modified to solve the problem on boating. The key to a successful solution is to have more people heading for Molokai than coming back to Oahu. Therefore, we must always have as many children at Molokai as possible to move the boat back to Oahu (which is the only way to make progress).
  
  All testing cases are contained in test function \texttt{Boat.selfTest()}. Since the structure of the problem is fixed, the only manipulatable parameters are the numbers of adults and children on board. Nevertheless, the number of adults really does not matter much, since they always cross the river one-at-a-time. The number of children is crucial only in the sense of its parity (but not its absolute value). On the whole, it is actually reasonable to only test two types of cases:
  \begin{itemize}
    \item An even number of children and several adults;
    \item An odd number of children and several adults.
  \end{itemize}    
  We also test on the edge case, where there are no adults.

\end{homeworkProblem}

% End edit to here
%%%%%%%%%%%%%%%%%%%%%%%%%%%%%%%%%%%%%%%%%%%%%%%%%%%%%%%%%%%%%

\end{spacing}
\end{document}

%%%%%%%%%%%%%%%%%%%%%%%%%%%%%%%%%%%%%%%%%%%%%%%%%%%%%%%%%%%%%
