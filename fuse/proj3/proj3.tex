%!TEX builder = latexmk
%!TEX program = xelatex
\documentclass{article}
\usepackage{enumitem}
\setenumerate[1]{leftmargin=2em, itemsep=0pt, partopsep=0pt, parsep=\parskip, topsep=0pt}
\setenumerate[2]{leftmargin=2em, itemsep=0pt, partopsep=0pt, parsep=\parskip, topsep=0pt}
\setitemize[1]{leftmargin=2em, itemsep=0pt, partopsep=0pt, parsep=\parskip, topsep=0pt}
\setitemize[2]{leftmargin=2em, itemsep=0pt, partopsep=0pt, parsep=\parskip, topsep=0pt}
\usepackage{amsmath,amsfonts,amsthm,amssymb}
\usepackage{setspace}
\usepackage{fancyhdr}
\usepackage{lastpage}
\usepackage{extramarks}
\usepackage{chngpage}
\usepackage{soul,color,framed}
\usepackage{graphicx,float,wrapfig}
\usepackage{CJK}
\usepackage{algorithm}  
\usepackage{algpseudocode} 
\usepackage{longtable}
\usepackage{listings}
\usepackage{color}
\usepackage{extarrows}
\usepackage{subfigure}
\usepackage{indentfirst}
\newcommand{\Class}{Operating System}
\newcommand{\ClassInstructor}{Wei Xu}

% Homework Specific Information. Change it to your own
\newcommand{\Title}{Project 3}
\newcommand{\DueDate}{Dec 13, 2020}
\newcommand{\StudentName}{}
\newcommand{\StudentClass}{}
\newcommand{\StudentNumber}{}

% In case you need to adjust margins:
\topmargin=-0.45in      %
\evensidemargin=0in     %
\oddsidemargin=0in      %
\textwidth=6.5in        %
\textheight=9.0in       %
\headsep=0.25in         %

% Setup the header and footer
\pagestyle{fancy}                                                       %
\lhead{\StudentName}                                                 %
\chead{\Title}  %
\rhead{\firstxmark}                                                     %
\lfoot{\lastxmark}                                                      %
\cfoot{}                                                                %
\rfoot{Page\ \thepage\ of\ \protect\pageref{LastPage}}                          %
\renewcommand\headrulewidth{0.4pt}                                      %
\renewcommand\footrulewidth{0.4pt}                                      %

%%%%%%%%%%%%%%%%%%%%%%%%%%%%%%%%%%%%%%%%%%%%%%%%%%%%%%%%%%%%%
% Some tools
\newcommand{\enterProblemHeader}[1]{\nobreak\extramarks{#1}{#1 continued on next page\ldots}\nobreak%
                                    \nobreak\extramarks{#1 (continued)}{#1 continued on next page\ldots}\nobreak}%
\newcommand{\exitProblemHeader}[1]{\nobreak\extramarks{#1 (continued)}{#1 continued on next page\ldots}\nobreak%
                                   \nobreak\extramarks{#1}{}\nobreak}%

\newcommand{\homeworkProblemName}{}%
\newcounter{homeworkProblemCounter}%
\newenvironment{homeworkProblem}[1][Problem \arabic{homeworkProblemCounter}]%
  {\stepcounter{homeworkProblemCounter}%
   \renewcommand{\homeworkProblemName}{#1}%
   \section*{\homeworkProblemName}%
   \enterProblemHeader{\homeworkProblemName}}%
  {\exitProblemHeader{\homeworkProblemName}}%

\newcommand{\homeworkSectionName}{}%
\newlength{\homeworkSectionLabelLength}{}%
\newenvironment{homeworkSection}[1]%
  {% We put this space here to make sure we're not connected to the above.

   \renewcommand{\homeworkSectionName}{#1}%
   \settowidth{\homeworkSectionLabelLength}{\homeworkSectionName}%
   \addtolength{\homeworkSectionLabelLength}{0.25in}%
   \changetext{}{-\homeworkSectionLabelLength}{}{}{}%
   \subsection*{\homeworkSectionName}%
   \enterProblemHeader{\homeworkProblemName\ [\homeworkSectionName]}}%
  {\enterProblemHeader{\homeworkProblemName}%

   % We put the blank space above in order to make sure this margin
   % change doesn't happen too soon.
   \changetext{}{+\homeworkSectionLabelLength}{}{}{}}%

\newcommand{\Answer}{\textbf{Answer:} }
\newcommand{\Acknowledgement}[1]{\ \\{\bf Acknowledgement:} #1}

%%%%%%%%%%%%%%%%%%%%%%%%%%%%%%%%%%%%%%%%%%%%%%%%%%%%%%%%%%%%%


%%%%%%%%%%%%%%%%%%%%%%%%%%%%%%%%%%%%%%%%%%%%%%%%%%%%%%%%%%%%%
% Make title
\title{\textmd{\bf \Class: \Title}\\{\large Instructed by \textit{\ClassInstructor}}\\\normalsize\vspace{0.1in}\small{Due\ on\ \DueDate}}
\date{}
\author{\textbf{\StudentName}\ \ \StudentClass\ \ \StudentNumber}
%%%%%%%%%%%%%%%%%%%%%%%%%%%%%%%%%%%%%%%%%%%%%%%%%%%%%%%%%%%%%


%%%%%%%%%%%%%%%%%%%%%%%%%%%%%%%%%%%%%%%%%%%%%%%%%%%%%%%%%%%%%
% Listing Settings
\definecolor{mygreen}{rgb}{0,0.6,0}
\definecolor{mygray}{rgb}{0.5,0.5,0.5}
\definecolor{mymauve}{rgb}{0.58,0,0.82}
\definecolor{shade}{rgb}{0.92,0.92,0.92}

\lstset{
  aboveskip=1em,                   % above skip space
  backgroundcolor=\color[rgb]{0.9,0.9,0.9},
                                   % choose the background color; you must add \usepackage{color} or \usepackage{xcolor}; should come as last argument
  basicstyle=\ttfamily,            % the size of the fonts that are used for the code
  breakatwhitespace=false,         % sets if automatic breaks should only happen at whitespace
  breaklines=true,                 % sets automatic line breaking
  captionpos=b,                    % sets the caption-position to bottom
  commentstyle=\color{mygreen},    % comment style
  deletekeywords={...},            % if you want to delete keywords from the given language
  escapeinside={\%*}{*)},          % if you want to add LaTeX within your code
  extendedchars=true,              % lets you use non-ASCII characters; for 8-bits encodings only, does not work with UTF-8
  frame=single,                    % adds a frame around the code
  keepspaces=true,                 % keeps spaces in text, useful for keeping indentation of code (possibly needs columns=flexible)
  keywordstyle=\color{blue},       % keyword style
  morekeywords={algexpr, frac, sqrt, pwr, b1, b2, b3, ln, sin, term},  
                                   % if you want to add more keywords to the set
  numbers=left,                    % where to put the line-numbers; possible values are (none, left, right)
  numbersep=5pt,                   % how far the line-numbers are from the code
  numberstyle=\color{mygray},      % the style that is used for the line-numbers
  rulecolor=\color{black},         % if not set, the frame-color may be changed on line-breaks within not-black text (e.g. comments (green here))
  showspaces=false,                % show spaces everywhere adding particular underscores; it overrides 'showstringspaces'
  showstringspaces=false,          % underline spaces within strings only
  showtabs=false,                  % show tabs within strings adding particular underscores
  stepnumber=5,                    % the step between two line-numbers. If it's 1, each line will be numbered
  stringstyle=\color{mymauve},     % string literal style
  tabsize=2,                       % sets default tabsize to 2 spaces
  title=\lstname,                  % show the filename of files included with \lstinputlisting; also try caption instead of title
  xleftmargin=2em,                 % left margin
  xrightmargin=2em,                % right margin
}
%%%%%%%%%%%%%%%%%%%%%%%%%%%%%%%%%%%%%%%%%%%%%%%%%%%%%%%%%%%%%


\renewcommand{\algorithmicrequire}{\textbf{Input:}}  % Use Input in the format of Algorithm  
\renewcommand{\algorithmicensure}{\textbf{Output:}} % Use Output in the format of Algorithm  


\newcommand*{\dif}{\mathop{}\!\mathrm{d}}
\newcommand*{\img}{\mathrm{i}}
\newcommand*{\dps}{\displaystyle}
\newcommand*{\lf}{\left\lfloor}
\newcommand*{\rf}{\right\rfloor}
\newcommand*{\lc}{\left\lceil}
\newcommand*{\rc}{\right\rceil}
\newcommand*{\ovt}[2]{\dps{\mathop{#1}^{#2}}}

\newtheorem{lma}{Lemma}

\newcommand{\lm}[1]{\textbf{Lemma \ref{#1}}}
\newcommand{\lgd}[2]{\left(\frac{#1}{#2}\right)}
\newcommand{\ds}[2]{\dps \frac{\partial #1}{\partial #2}}
\newcommand{\bds}[2]{\dps \frac{\partial}{\partial #2}\left(#1\right)}
\newcommand{\Eds}[2]{\dps \frac{\partial^2 #1}{\partial #2^2}}
\newcommand{\eds}[3]{\dps \frac{\partial^2 #1}{\partial #2\partial #3}}
\newcommand{\E}[1]{\mathbb{E}\left[#1\right]}
\newcommand{\Var}[1]{\mathrm{Var}\left[#1\right]}
\newcommand{\Cov}[1]{\mathrm{Cov}\left[#1\right]}

\newcommand{\Fig}[1]{\textbf{Figure \ref{#1}}}

\begin{document}
\begin{spacing}{1.1}
\maketitle \thispagestyle{empty}
%\cite{}
%%%%%%%%%%%%%%%%%%%%%%%%%%%%%%%%%%%%%%%%%%%%%%%%%%%%%%%%%%%%%
% Begin edit from here

\begin{homeworkProblem}[Data Structures]
\end{homeworkProblem}

\begin{homeworkProblem}[Code]
\end{homeworkProblem}

\begin{homeworkProblem}[Details for Requirements 7-10]

\begin{itemize}

\item \emph{Requirement 7:} \texttt{chmod} and \texttt{chown} by
modifying metadata in inodes.

\item \emph{Requirement 8:} A global lock is declaired in
\colorbox{shade}{\texttt{lfs/utility.cpp}}. For simplicity in project 3, all
functions require this lock when start and return this lock before return.

\item \emph{Requirement 9:} We implemented a segment buffer, and flush it
whenever \texttt{sync} is called.

\item \emph{Requirement 10:} \texttt{init} function in
\colorbox{shade}{\texttt{lfs/system.cpp}} scans the permanent storage file on
disk, and restores the file system before crash. It only takes into account
segments which have been successfully written to disk, which can be determined by
extra timestamps recorded in each segment. Inode maps can be restored by reading
segment summary and imap table in each segment.

\end{itemize}

\end{homeworkProblem}

\begin{homeworkProblem}[Tests]

  Open terminal in \colorbox{shade}{\texttt{proj3}} directory, then run 
  \colorbox{shade}{\texttt{bash test.sh}}. This will execute all five tests.
  Refer to \colorbox{shade}{\texttt{test.sh}} for compile options if you
  want to run each test separately.

  \noindent\fbox{\colorbox{shade}{\textbf{\ttfamily testfile.cpp}}}
  
  A test for operations \texttt{open}, \texttt{create} and \texttt{write}.
  This test creates a file with $100$ characters.
  Copy the binary file to an empty directory and execute 
  \colorbox{shade}{\texttt{./testfile}}.

  If the file system functions properly, there should be no errors.

  \noindent\fbox{\colorbox{shade}{\textbf{\ttfamily testfile2.cpp}}}
  
  A test for operations \texttt{create}, \texttt{write} and \texttt{mkdir},
  an implicit requirement is thread-safety.
  This test creates $1000$ directories, each with a file inside.
  Copy the binary file to an empty directory and execute 
  \colorbox{shade}{\texttt{./testfile2}}.

  If the file system functions properly, there should be no errors.

  \noindent\fbox{\colorbox{shade}{\textbf{\ttfamily testmkdir.cpp}}}
  
  A stress test for block-segment management and operation \texttt{mkdir}.
  This test creates directories named $0,1,2,\ldots,n-1$.
  Copy the binary file to an empty directory and execute 
  \colorbox{shade}{\texttt{./testmkdir <n>}}.

  If the file system functions properly, there should be no errors.
  
  \noindent\fbox{\colorbox{shade}{\textbf{\ttfamily testrmdir.cpp}}}
  
  A stress test for block-segment management and operation \texttt{rmdir}.
  This test creates a tree structure of $n$ directories first, then keeps removing
  a random directory until all directories are deleted.
  Copy the binary file to an empty directory and execute 
  \colorbox{shade}{\texttt{./testrmdir <n>}}.

  If the file system functions properly, \texttt{testrmdir} should not
  exit due to assertion failure.

  \noindent\fbox{\colorbox{shade}{\textbf{\ttfamily testconcurrency.cpp}}}
  
  A stress test for block-segment management and thread-safety.
  This test invokes $n$ threads. Each thread creates $m$ directories, each
  with a file inside. Copy the binary file to an empty directory and execute 
  \colorbox{shade}{\texttt{./testrmdir <n> <m>}}.

  If the file system functions properly, there should be exactly
  $n\times m$ directories.
\end{homeworkProblem}

\begin{homeworkProblem}[Manual]

\begin{itemize}

\item To compile, execute \colorbox{shade}{\texttt{scons}} in 
\colorbox{shade}{\texttt{lfs/}} directory. If any issue happens, you may need
to use Ubuntu 20.04 and install scons.

\item To mount file system, either execute \colorbox{shade}{\texttt{bash buildfs.sh}}
in \colorbox{shade}{\texttt{proj3/}} directory, or manually execute
(this also applies to the ``echo file system'' in task 1)
\colorbox{shade}{\texttt{./fuse <mount directory> <options>}}.

\item Line 190-204 of \colorbox{shade}{\texttt{lfs/utility.h}} contain several
debug options. To enable ``echo'' in log-structured file system (should add
\colorbox{shade}{\texttt{-f}} to options), toggle on \texttt{DEBUG\_PRINT\_COMMAND}.

\item Line 213 of \colorbox{shade}{\texttt{lfs/utility.h}} is a switch for 
directory access time updates. When \texttt{FUNC\_ATIME\_DIR} is $1$, access
timestamps of all files along the path will be updated.

\end{itemize}

\end{homeworkProblem}

\begin{homeworkProblem}[Limitations]

We have not implemented garbage collection, so when the file system is full, we
can not even delete files.

\end{homeworkProblem}

% End edit to here
%%%%%%%%%%%%%%%%%%%%%%%%%%%%%%%%%%%%%%%%%%%%%%%%%%%%%%%%%%%%%
\end{spacing}
\end{document}

%%%%%%%%%%%%%%%%%%%%%%%%%%%%%%%%%%%%%%%%%%%%%%%%%%%%%%%%%%%%%
